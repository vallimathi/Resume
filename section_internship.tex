\sectionTitle{Research Internship}{\faPaw}

\textbf{Active Structures : Design and Control} \hfill{May 2019 - July 2019}\\
\textbf{Guide: Prof. Ian F. C. Smith, Dr. Gennaro Senatore, IMAC, ENAC} \hfill{EPFL, Switzerland} \\
\textbf{Description:} {Adaptive structures undergo real-time shape changes to accommodate the loads acting on them. The project was focused on developing the real-time control problem and assembly of a full-scale prototype bridge.}
\begin{itemize}
    \setlength\itemsep{0.7mm}
    \item Incorporated the nodal coordinates of structure in the set of design variables of existing algorithms for shape optimization.
    \item Formulated a physically significant objective function and derived its Hessian and Jacobian Matrices to establish the limits of convexity and the existence of local and global minima of the problem.
    \item Designed and implemented an iterative linearized control problem to achieve time efficiency required for real-time control.
    \item Derived the closed-form solution of the multi-variate non-linear control problem using Karush-Kuhn-Tucker (KKT) optimality criterion and symmetry conditions present in the structure.
    \item Results from non-linear optimization using Simple Quadratic Programming (SQP) and Interior-point method (IPM), iterative optimization of the linearized problem and closed-form solution of problem using KKT optimality criterion agreed closely.
    \item Helped in the assembly of a 6-bay adaptive prototype full-scale pedestrian bridge, gained insights into the working of strain sensors and CAN bus protocols that handle the communication between the computing unit and actuators.
\end{itemize}